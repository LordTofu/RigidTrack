\hypertarget{index_intro_sec}{}\section{Introduction}\label{index_intro_sec}
Rigid Track is a software that provides, combined with an Opti\+Track camera, the pose estimation of one object in three dimensional space. This is achieved with only one camera in combination with reflective markers. Those are attached to the object ought to be tracked. The accuracy in the range of millimeters and the high update rate of 100 Hz enable use cases for fast and agile objects. The main application is navigation for drones that rely on high precision position data. Where G\+PS is not available, e.\+g. indoors or due to a lacking G\+PS receiver, this setup substitutes for it. Another use case is the pure pose logging when the drone does not depend on the position, e.\+g. when it is remote piloted by hand. While this setup contains one Opti\+Track Flex 3 camera, every other model of Opti\+Track should work, despite not tested. With better camera models, e.\+g. the Prime Series, even outdoor usage is possible. When the capabilities are not sufficient please refer to Opti\+Tracks Software Motive. But keep in mind that this solution needs at least 3 cameras as Rigid Track works with only one.\hypertarget{index_softInstall_sec}{}\section{Rigid Track Installation}\label{index_softInstall_sec}
Start the Rigid\+Track\+\_\+setup.\+exe from the enclosed SD card and follow the instructions given in the installation assistant. Default parameters like installation directory or shortcuts to be created can be chosen. But normally clicking Next and keeping the default values should be sufficient. When the installation is completed a shortcut in the start menu and the desktop can be used to start Rigid Track. The program is then successfully installed in C\+:/\+Program Files (x86)/\+TU Munich F\+S\+D/\+Rigid Track.\hypertarget{index_source_code}{}\section{Source Code}\label{index_source_code}
The most interesting file for you is \hyperlink{main_8cpp}{main.\+cpp}. It contains the relevant functions for pose estimation. Camera calibration and other functional aspects are also implemented there. The G\+UI program code is found in \hyperlink{_rigid_track_8cpp}{Rigid\+Track.\+cpp}. \hyperlink{communication_8cpp}{communication.\+cpp} deals only with communication from \hyperlink{main_8cpp}{main.\+cpp} to the G\+UI. 